\documentclass[11pt]{article}
\usepackage{geometry}                
\geometry{letterpaper}                   

\usepackage{graphicx}
\usepackage{amssymb}
\usepackage{epstopdf}
\usepackage{natbib}
\usepackage{amssymb, amsmath}
\DeclareGraphicsRule{.tif}{png}{.png}{`convert #1 `dirname #1`/`basename #1 .tif`.png}

%\title{Title}
%\author{Name 1, Name 2}
%\date{date} 

\begin{document}


\input{cover}
\newpage

%%%%%%%%%%%%%%%%%%%%%%%%%%%%%%%%%%%%%%%%%%%%%%%%%

\newpage
\section*{Agreement for free-download}
\bigskip


\bigskip


\large We hereby agree to make our source code for this project freely available for download from the web pages of the SOMS chair. Furthermore, we assure that all source code is written by ourselves and is not violating any copyright restrictions.

\begin{center}

\bigskip


\bigskip


\begin{tabular}{@{}p{3.3cm}@{}p{6cm}@{}@{}p{6cm}@{}}
\begin{minipage}{3cm}

\end{minipage}
&
\begin{minipage}{6cm}
\vspace{2mm} \large Name 1

 \vspace{\baselineskip}

\end{minipage}
&
\begin{minipage}{6cm}

\large Name 2

\end{minipage}
\end{tabular}


\end{center}
\newpage

%%%%%%%%%%%%%%%%%%%%%%%%%%%%%%%%%%%%%%%



% IMPORTANT
% you MUST include the ETH declaration of originality here; it is available for download on the course website or at http://www.ethz.ch/faculty/exams/plagiarism/index_EN; it can be printed as pdf and should be filled out in handwriting


%%%%%%%%%% Table of content %%%%%%%%%%%%%%%%%

\tableofcontents

\newpage

%%%%%%%%%%%%%%%%%%%%%%%%%%%%%%%%%%%%%%%



\section{Abstract}

\section{Individual contributions}

\section{Introduction and Motivations}

\section{Description of the Model}

\section{Implementation}
\subsection{Interaction forces}
\subsubsection{Pedestrian interaction force}
There are two interaction forces that are calculated separately for every pedestrian in every timestep. \\
The first one is the so called pedestrian interaction force, which is initialized as a repulsive force and and therefore allows each pedestrian to keep a certain distance to all the other pedestrians. \\
In our simulation, the repulsive pedestrian interaction force has been specified according to the formula
%
\begin{equation}
\vec{f}_{\alpha\beta}(t) = A_\alpha ^1 e^{\frac{r_{\alpha\beta} -d_{\alpha\beta}}{B_\alpha ^1}}\vec{n}_{\alpha\beta}
\cdot{\left(\lambda _\alpha + (1-\lambda _\alpha)
\frac{1+\cos{\phi_{\alpha\beta}}}{2}\right)}
+ A_\alpha ^2 e^{\frac{r_{\alpha\beta} -d_{\alpha\beta}}{B_\alpha ^2}}\vec{n}_{\alpha\beta}
\end{equation}
%
,where 
\(d_{\alpha\beta}(t)=\left\|\vec{x}_\alpha (t)-\vec{x}_\beta (t)\right\|\)
is the distance between the two pedestrians $\alpha$ and $\beta$, \(r_{\alpha\beta}=(r_\alpha + r_\beta)\) is the sum of their radii and 
\(\vec{n}_{\alpha\beta}(t)=[\vec{x}_\alpha(t)-\vec{x}_\beta(t)]/d_{\alpha\beta}\)
is the normalized vector from $\beta$ to $\alpha$.
\( \phi_{\alpha\beta}(t) \) is the angle between \(\vec{n}_{\alpha\beta}(t)\) and the direction of movement \( \vec{e}_\alpha(t)=\vec{v}_\alpha(t)/\left\|\vec{v}_\alpha(t)\right\| \) of the pedestrian $\alpha$.

\section{Simulation Results and Discussion}

\section{Summary and Outlook}

\section{References}






\end{document}  



 
